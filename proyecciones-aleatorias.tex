\batchmode
\makeatletter
\def\input@path{{/home/gonzalo/maestria/proyecciones-aleatorias/}}
\makeatother
\documentclass[10pt,spanish,handout]{beamer}\usepackage[]{graphicx}\usepackage[]{color}
% maxwidth is the original width if it is less than linewidth
% otherwise use linewidth (to make sure the graphics do not exceed the margin)
\makeatletter
\def\maxwidth{ %
  \ifdim\Gin@nat@width>\linewidth
    \linewidth
  \else
    \Gin@nat@width
  \fi
}
\makeatother

\definecolor{fgcolor}{rgb}{0.345, 0.345, 0.345}
\newcommand{\hlnum}[1]{\textcolor[rgb]{0.686,0.059,0.569}{#1}}%
\newcommand{\hlstr}[1]{\textcolor[rgb]{0.192,0.494,0.8}{#1}}%
\newcommand{\hlcom}[1]{\textcolor[rgb]{0.678,0.584,0.686}{\textit{#1}}}%
\newcommand{\hlopt}[1]{\textcolor[rgb]{0,0,0}{#1}}%
\newcommand{\hlstd}[1]{\textcolor[rgb]{0.345,0.345,0.345}{#1}}%
\newcommand{\hlkwa}[1]{\textcolor[rgb]{0.161,0.373,0.58}{\textbf{#1}}}%
\newcommand{\hlkwb}[1]{\textcolor[rgb]{0.69,0.353,0.396}{#1}}%
\newcommand{\hlkwc}[1]{\textcolor[rgb]{0.333,0.667,0.333}{#1}}%
\newcommand{\hlkwd}[1]{\textcolor[rgb]{0.737,0.353,0.396}{\textbf{#1}}}%
\let\hlipl\hlkwb

\usepackage{framed}
\makeatletter
\newenvironment{kframe}{%
 \def\at@end@of@kframe{}%
 \ifinner\ifhmode%
  \def\at@end@of@kframe{\end{minipage}}%
  \begin{minipage}{\columnwidth}%
 \fi\fi%
 \def\FrameCommand##1{\hskip\@totalleftmargin \hskip-\fboxsep
 \colorbox{shadecolor}{##1}\hskip-\fboxsep
     % There is no \\@totalrightmargin, so:
     \hskip-\linewidth \hskip-\@totalleftmargin \hskip\columnwidth}%
 \MakeFramed {\advance\hsize-\width
   \@totalleftmargin\z@ \linewidth\hsize
   \@setminipage}}%
 {\par\unskip\endMakeFramed%
 \at@end@of@kframe}
\makeatother

\definecolor{shadecolor}{rgb}{.97, .97, .97}
\definecolor{messagecolor}{rgb}{0, 0, 0}
\definecolor{warningcolor}{rgb}{1, 0, 1}
\definecolor{errorcolor}{rgb}{1, 0, 0}
\newenvironment{knitrout}{}{} % an empty environment to be redefined in TeX

\usepackage{alltt}
\usepackage{amstext}
\usepackage{amssymb}
\usepackage{fontspec}
\usepackage{mathtools}
\ifx\hypersetup\undefined
  \AtBeginDocument{%
    \hypersetup{unicode=true}
  }
\else
  \hypersetup{unicode=true}
\fi

\makeatletter
%%%%%%%%%%%%%%%%%%%%%%%%%%%%%% Textclass specific LaTeX commands.
% this default might be overridden by plain title style
\newcommand\makebeamertitle{\frame{\maketitle}}%
% (ERT) argument for the TOC
\AtBeginDocument{%
  \let\origtableofcontents=\tableofcontents
  \def\tableofcontents{\@ifnextchar[{\origtableofcontents}{\gobbletableofcontents}}
  \def\gobbletableofcontents#1{\origtableofcontents}
}

%%%%%%%%%%%%%%%%%%%%%%%%%%%%%% User specified LaTeX commands.
\usetheme[block=fill]{metropolis}
%\useinnertheme{metropolis}
%\useoutertheme{metropolis}
%\usefonttheme{metropolis}
%\usecolortheme{metropolis}
%\usetheme{Darmstadt}
%\usetheme{Frankfurt}
% or ...

%\setbeamercovered{transparent}

\makeatother

\usepackage{polyglossia}
\setdefaultlanguage{spanish}
\IfFileExists{upquote.sty}{\usepackage{upquote}}{}
\begin{document}
\global\long\def\H{\mathbb{H}}%


\title[Proyecciones Aleatorias]{Proyecciones en Direcciones Aleatorias}
\subtitle{o “Cómo reducir dimensionalidad sin pensar demasiado”}
\author{Gonzalo Barrera Borla }
\institute{Universidad de Buenos Aires}
\date{Análisis de Datos Funcionales, 2020}

\makebeamertitle

%\beamerdefaultoverlayspecification{<+->}

%\AtBeginSubsection[]{%
%  \frame<beamer>{ 
%    \frametitle{Outline}   
%   \tableofcontents[currentsection,currentsubsection] 
% }
%}

%\setbeamertemplate{note page}[plain]
%\setbeameroption{show notes}
\setbeamertemplate{section in toc}[sections numbered]
\begin{frame}{Contenido}

\tableofcontents{}
\end{frame}

\section{Motivación}

\subsection{El problema: caracterizar e. a. $\infty-\text{dimensionales}$}
\begin{frame}{El problema: caracterizar e. a. $\infty-\text{dimensionales}$}

Caracterizar elementos aleatorias en espacios $\infty$-dimensionales
implica expresarlos (i.e. proyectarlos) en \emph{algún} subespacio
de $\H$:

\pause{}
\begin{itemize}
\item bases fijas: Fourier, b-splines, ...
\end{itemize}

\pause{}
\begin{itemize}
\item descomposiciones data-dependientes: FPC/KLT, FCC, ...
\end{itemize}

\pause{}
\begin{itemize}
\item o en direcciones aleatorias.
\end{itemize}

\pause{}

¿Guarda alguna información útil una proyección aleatoria?

\pause{}

Resulta que sí, y bastante.
\end{frame}

\subsection{Una herramienta más: direcciones aleatorias}
\begin{frame}{Una herramienta más: direcciones aleatorias}

Sean
\begin{itemize}
\item $X,Y$ elementos aleatorios (e. a.) en el espacio de Hilbert $\H,\ X\sim P,\ Y\sim Q$,
\end{itemize}

\pause{}
\begin{itemize}
\item $P\in DM\left(\H\right)$, una distribución \emph{determinada por
sus momentos}, y
\end{itemize}

\pause{}
\begin{itemize}
\item $\mathbb{E}$ el cono cerrado $\mathbb{E}\left(X,Y\right)\coloneqq\left\{ \alpha\in\mathbb{H}:\left\langle X,\alpha\right\rangle \sim\left\langle Y,\alpha\right\rangle \right\} $
\end{itemize}

\pause{}
\begin{exampleblock}{Teorema Cuesta - Fraiman - Ransford}

(Adaptado de \cite{Cuesta 2007}) Sea $\mu$ una distribución Gaussiana
no-degenerada en $\H$. Luego, 
\[
X\sim Y\iff\mu\left[\mathbb{E}\left(X,Y\right)\right]>0
\]
.
\end{exampleblock}
\end{frame}
\note{Alguna información siempre van a guardar, la pregunta es qué se puede
decir a partir de ella. Resulta que bastante

E el cono determinado por las direcciones alfa en las que las distribuciones
marginales de X e Y conciden

Es un resultado bastante fuerte. Eligiendo direcciones $\alpha$ al
azar según cierta distribución sencilla, si la distribución de las
proyecciones de X e Y en alfa coinciden, casi-seguramente coinciden
las distribuciones de X e Y.}

\section[Aplicaciones]{Algunas aplicaciones}

\subsection[Profundidad aleatoria]{Noción de Profundidad}
\begin{frame}{Profundidad aleatoria proyectada}

Existen distintas nociones de profundad posibles posibles en $\H$:

\pause{}
\begin{itemize}
\item profundidad \emph{integrada} de Fraiman y Muniz,
\end{itemize}

\pause{}
\begin{itemize}
\item el método de la \emph{h-moda},
\end{itemize}

\pause{}
\begin{itemize}
\item en direcciones aleatorias.
\end{itemize}
\end{frame}
%
\begin{frame}{Profundidad aleatoria proyectada}

Dada una muestra $\mathbf{X}=\left\{ X_{i}\right\} _{i=1}^{n}$, sean
\begin{itemize}
\item $\alpha\in\H$ una dirección aleatoria independiente de $\mathbf{X}$
y
\end{itemize}

\pause{}
\begin{itemize}
\item $\mathbf{x}=\left\{ x_{i}:x_{i}=\left\langle X_{i},\alpha\right\rangle \right\} _{i=1}^{n}$
la proyección de cada elemento de $\mathbf{X}$ en ella.
\end{itemize}

\pause{}

La \emph{profundidad aleatoria proyectada }muestral del dato original
$\mathop{RPD}\left(X_{i}\right)$ se define como la profundidad unidimensional
de $x_{i}$, expresada en términos del cuantil \emph{aleatorio} de
$x_{i}$ en $\mathbf{x}$.

\pause{}
\begin{alertblock}{¡Ojo!}

Al ser una medida aleatoria, hay que controlar su variabilidad, típicamente
promediando sobre múltiples proyecciones $\left\{ \alpha_{i}\right\} _{i=1}^{n}$.
\end{alertblock}

\pause{}

Estas medidas de profundidad se pueden usadas como covariables en
tareas de regresión \cite{Cuevas 2007} y y clasificación \cite{Cuevas 2007,Cuesta 2017}.
\end{frame}

\subsection{Test de hipótesis vía proyecciones:}

\subsubsection{Bondad de ajuste}
\begin{frame}{Bondad de ajuste para familias paramétricas}

Consideremos la hipótesis de pertenencia a cierta familia de distribuciones
paramétricas

\begin{align*}
H_{0} & :P\in\mathcal{P}\coloneqq\left\{ P\left(\cdot,\theta\right):\theta\in\Theta\right\} \quad\text{vs.}\quad H_{1}:P\notin\mathcal{P}\\
\end{align*}


\pause{}

¿Cómo se usaría el teorema de Cuesta-Fraiman-Ransford para construir
tests a partir de proyecciones aleatorias?
\end{frame}
%
\begin{frame}{Bondad de ajuste para familias paramétricas}

\cite{Cuesta 2007} considera familias paramétricas
\begin{enumerate}[<+->]
\item invariantes
\begin{enumerate}[<+->]
\item por locación (\emph{l-}invariantes\emph{)}: $X\sim P_{X}\in\mathcal{P},Y=X+m\Rightarrow P_{y}\in\mathcal{P}$
\item por escala (\emph{s-}invariantes\emph{)}: $X\sim P_{X}\in\mathcal{P},Y=\Sigma X\Rightarrow P_{y}\in\mathcal{P}$
\item elípticas (\emph{l- y s- }invariantes)
\end{enumerate}
\item \emph{ciertas }familias no-invariantes, donde
\begin{itemize}
\item $\mathcal{P}\coloneqq\left\{ \mathcal{L}\left(sX+g\right):s\in\mathbb{R}\land g\in V_{n}\right\} $
y
\item $V_{n}$ es un subespacio finito-dimensional de $\mathbb{H}$
\end{itemize}
\end{enumerate}

\pause{}

y prueba que familias del tipo
\begin{enumerate}[<+->]
\item están \emph{determinadas en probabilidad} (d. p.) por una única proyección
aleatoria.
\item con $n$ parámetros están d. p. por $k=n+1$ proyecciones aleatorias.
\end{enumerate}
\end{frame}
%
\begin{frame}{Bondad de ajuste para familias paramétricas}

En cualquier caso, si la familia determinada por $\left\{ h_{i}\right\} _{i=1}^{k}$
proyecciones aleatorias $\mu-\text{generadas}$, resulta que la \emph{hipótesis
proyectada}

\vspace{-15pt}

\begin{align*}
H_{0}^{h} & :\left(P_{h_{1}},\ldots,P_{h_{k}}\right)\in\mathcal{P}_{h_{1},\ldots,h_{k}}:=\left\{ \left(P_{h_{1}}(\cdot,\theta),\ldots,P_{h_{k}}(\cdot,\theta)\right):\theta\in\Theta\right\} 
\end{align*}


\pause{}

es $\mu^{k}-\text{casi seguramente}$($\mu^{k}-c.s.$) equivalente
a la hipótesis original, ya que $\mu^{k}-c.s.$

\vspace{-15pt}

\begin{align*}
\left(P_{h_{1}},\ldots,P_{h_{k}}\right)\in\mathcal{P}_{h_{1},\ldots,h_{k}} & \iff P\in\mathcal{P}\\
H_{0}^{h} & \iff H_{0}
\end{align*}


\pause{}

Y para alguna medida apropiada de distancia $d$ que mida el desvío
de $P$ c.r.a. $\mathcal{P}$ bajo $H_{0}$, $\mu^{k}-c.s.$

\vspace{-15pt}

\[
\max_{i=1,\ldots,k}d\left(P_{h_{i}},P_{h_{i}}(\cdot,\theta)\right)=0
\]

\end{frame}
%
\begin{frame}{Bondad de ajuste para familias paramétricas}

Tomando por $d$ la distancia Kolmogoro-Smirnov, podemos basar el
test en el estadístico

\[
D_{n}:=\max_{i=1,\ldots,k}\sqrt{n}d\left(\left(\mathbb{P}_{n}\right)_{h_{i}},P_{h_{i}}(\cdot,\theta)\right)
\]

i.e. el máximo de $k$estadísticos K-S univariados, donde $\mathbb{P}_{n}$es
la distribución empírica basada en $\left\{ X_{i}\right\} _{i=1}^{n}$.

\pause{}

Restan dos dificultades:
\begin{itemize}
\item Cuando $k=1$, el estadístico K-S es de distribución libre, pero para
$k>1$ ya no.
\end{itemize}

\pause{}
\begin{itemize}
\item Salvo para hipótesis “simples” $H_{0}:P_{0}=P\left(\cdot,\theta_{0}\right)$,
es necesario estimar $\theta$ a partir de $\mathbf{X}$.
\end{itemize}

\pause{}

Así que tendremos que contentarnos con un estimador $\hat{D_{n}}$
de $D_{n}$. Para aproximar la distribución de $\hat{D_{n}}$, sugieren
un procedimiento bootstrap y prueban que vale bajo ciertas condiciones.
\end{frame}
%

\subsubsection{MLF con respuesta escalar}
\begin{frame}{MLF con respuesta escalar}

\cite{Garcia 2014,Cuesta 2019} aplican estas ideas al modelo lineal
funcional con respuesta escalar

\[
Y=\left\langle X,\beta\right\rangle +\varepsilon=\int X\left(t\right)\beta\left(t\right)dt+\varepsilon
\]

donde $X,\beta\in\H,Y\in\mathbb{R}$ y se sostienen los supuestos
clásicos sobre $\epsilon$ detallados en los apuntes de clase.

\pause{}

La predicción de $Y$ se hace a partir de la esperanza condicional

\[
m\left(X\right)=E\left[Y|X\right]=\left\langle X,\beta\right\rangle 
\]


\pause{}

de manera que las siguientes expresiones son equivalentes:
\begin{itemize}
\item El MLF representa adecuadamente la relación entre $X$ e $Y$
\item $m\left(X\right)\in\mathcal{M}\coloneqq\left\{ \left\langle \cdot,\beta\right\rangle :\beta\in\H\right\} $
\end{itemize}
\end{frame}
%
\begin{frame}{MLF con respuesta escalar}

A partir de aquí, se pueden utilizar las técnicas antes mencionadas
para construir tests
\begin{enumerate}[<+->]
\item para hipótesis simples : $\beta=\beta_{0}$
\item incluido el caso particular de “no interacción”, $\beta_{0}\left(t\right)=0\ \forall\ t$,
y
\item para la significatividad global, estimando previamente $\beta$ 
\end{enumerate}
\uncover<+->{La clave, está en el siguiente lema, adaptado de \cite{Patilea 2012},
que permite caracterizar $H_{0}:m\in\mathcal{M}$.}
\end{frame}
%
\begin{frame}{MLF con respuesta escalar}

Consideremos
\begin{itemize}[<+->]
\item $\H=L^{2}\left[a,b\right]$
\item $\left\{ \Psi_{j}\right\} _{j=1}^{\infty}$una base de $\H$, no necesariamente
ortogonal,
\item la \emph{esfera funcional} $\mathbb{S_{H}}\left\{ h\in\H:\left\Vert h\right\Vert _{\H}=1\right\} $,
y
\item la \emph{esfera funcional p-dimensional $\mathbb{S}_{\H}^{p}=\left\{ h=\sum_{j=1}^{p}x_{j}\Psi_{j}\in\mathbb{H}:\|h\|_{\mathbb{H}}=1\right\} $. }
\end{itemize}
\begin{exampleblock}<+->{Lema \cite{Patilea 2012}}

Sea $\beta$ un elemento de $\H$. Las siguientes afirmaciones son
equivalentes:
\begin{enumerate}[<+->]
\item $m\left(X\right)=\left\langle X,\beta\right\rangle \forall X\in\H$
\item $\mathbb{E}\left[Y-\left\langle X,\beta\right\rangle |X=x\right]=0$
para casi todo (p.c.t.) $x\in\H$
\item $\mathbb{E}\left[Y-\left\langle X,\beta\right\rangle |\left\langle X,\gamma\right\rangle =u\right]=0$
p.c.t. $u\in\mathbb{R}$ y $\forall\gamma\in\mathbb{S_{H}}$
\item $\mathbb{E}\left[Y-\left\langle X,\beta\right\rangle |\left\langle X,\gamma\right\rangle =u\right]=0$
p.c.t. $u\in\mathbb{R}$ y $\forall\gamma\in\mathbb{S}_{\H}^{p},\forall p\geq1$
\end{enumerate}
\end{exampleblock}
\end{frame}
\note{Este resultado es realmente fuerte: no sólo alcanza con una dirección
aleatoria para caracterizar la hipótesis nula, sino que dicha dirección
puede provenir de un subespacio de dimensión finita, siempre y cuando
éste se elija con mínimos cuidados.}

\section[ANOVA funcional]{ANOVA funcional }

\subsection{Descripción del test}
\begin{frame}{Descripción del test}

\cite{Cuesta 2010} propone un procedimiento relativamente sencillo
para diseños con dos factores, interacciones y covariables funcionales.

\pause{}

Sean
\begin{itemize}
\item $\H$ un espacio de Hilbert separable con producto interno $\left\langle \cdot,\cdot\right\rangle $
medido w.l.g. en el intervalo $\left[0,1\right]$, y
\item $R,S\in\mathbb{N}:\forall\left(r,s\right)\in\left\{ 1,\dots,R\right\} \times\left\{ 1,\dots,S\right\} $ 
\end{itemize}

\pause{}

Y existen $X_{i}^{r,s},i\in\left\{ 1,\dots,n_{r,s}\right\} $ e. a.
en $\H$ tales que

\vspace{-15pt}

\[
X_{i}^{r,s}\left(t\right)=m\left(t\right)+f^{r}\left(t\right)+g^{s}\left(t\right)+h^{r,s}\left(t\right)+\gamma\left(t\right)Y_{i}^{r,s}+\epsilon_{i}^{r,s}\left(t\right),\quad t\in\left[0,1\right]
\]

\end{frame}
%
\begin{frame}{Descripción del test}

\[
X_{i}^{r,s}\left(t\right)=m\left(t\right)+f^{r}\left(t\right)+g^{s}\left(t\right)+h^{r,s}\left(t\right)+\gamma\left(t\right)Y_{i}^{r,s}+\epsilon_{i}^{r,s}\left(t\right),\quad t\in\left[0,1\right]
\]


\pause{}

donde
\begin{enumerate}[<+->]
\item la función $m$ es fija y describe la forma general del proceso,
\item las funciones fijas $f^{r},g^{s},h^{r,s}\in\H$ representan, respectivamente,
el efecto del primer y segundo factor, y la interacción entre ambos.\note{Como en todo diseño, es necesario hacer algún tipo de supuestos de
identificabilidad \emph{alla} $\sum_{r}f^{r}\left(t\right)=0$.}
\item Las $Y_{i}^{r,s}\in\mathbb{R}$ son cantidades aleatorias y conocidas
que influyen el proceso según el peso de la función fija y conocida
$\gamma\in\H$.\note{\begin{enumerate}
\item Las $Y_{i}^{r,s}$juegan el papel de las covariables en el caso unidimensional.
\end{enumerate}
}
\item Las trayectorias aleatorias $\epsilon_{i}^{r,s}\left(t\right)\in\H$
se asumen independientes y centradas. Además, para cada par $\left(r,s\right)$
fijo, $\epsilon_{i}^{r,s}\sim\text{iid}\ \forall\ i\in\left\{ 1,\dots,n_{r,s}\right\} $
\end{enumerate}
\end{frame}
%
\begin{frame}{Descripción del test}

Las hipótesis de interés son:

\vspace{-15pt}

\begin{alignat*}{2}
H_{0}^{A}: & f^{1}=\dots=f^{R}=0 & \textnormal{(el primer factor no tiene efecto)}\\
H_{0}^{B}: & g^{1}=\dots=g^{S}=0 & \textnormal{(el segundo factor no tiene efecto)}\\
H_{0}^{I}: & h^{1,1}=\dots=h^{R,S}=0 & \textnormal{(no hay interacción)}\\
H_{0}^{C}: & \gamma=0 & \textnormal{(la covariable no es significativa)}
\end{alignat*}


\pause{}

Consideraremos especialmente $H_{0}^{A}.$ Por CFR, si existen $r_{1,}r_{2}:f^{r_{1}}\neq f^{r_{2}}$,
para cualquier medida $\mu$ gaussiana no-degenerada en $\H$,
\[
\mu\left\{ v\in\H:\left\langle v,f^{1}\right\rangle =\dots=\left\langle v,f^{R}\right\rangle \right\} =0
\]


\pause{}

Sea $v$ un elemento elegido al azar según $\mu$. Luego, condicional
a que $H_{0}^{A}$ se cumple, para cada $v\in\H$

\vspace{-15pt}

\[
H_{0}^{A,v}:\left\langle v,f^{1}\right\rangle =\dots=\left\langle v,f^{R}\right\rangle =0
\]

también se cumple, y si $H_{0}^{A}$ no se cumple, $\mu-\textnormal{c.s.}\ H_{0}^{A,v}$
tampoco.
\end{frame}

\subsection{Consideraciones prácticas}
\begin{frame}{Consideraciones prácticas}
\begin{alertblock}<+->{Supuestos del modelo}
\begin{itemize}[<+->]
\item Al igual que en ANOVA clásico, las hipótesis de homocedasticidad
y/o gaussianidad de los datos son cruciales. Para elegir qué test
ANOVA aplicar, podemos analizar las proyecciones aleatorias. 
\item En funciones aleatorias, la homocedasticidad no es supuesto razonable,
ya que las oscilaciones del proceso suelen depender de sus valores,
así que conviene tener a manor un test ANOVA unidimensional que funcione
bien bajo condiciones de heterocedasticidad.
\end{itemize}
\end{alertblock}

\pause{}
\begin{alertblock}<+->{Potencia y estabilidad}
\begin{itemize}[<+->]
\item El reemplazo de una función $\in\H$ por un único número real acarrea
pérdida de información, y por ende de potencia para detectar alternativas
\item Al estar basado en una proyección elegida \emph{al azar}, de repetir
el procedimiento dos o más veces podemos obtener resultados diferentes.
\end{itemize}
\end{alertblock}
\end{frame}
%
\begin{frame}{False Discovery Rate}

Para reducir estos inconvenientes, se pueden tomar $k>1$ proyecciones
aleatorias, testear $H_{0}$ bajo c.u. y combinar los \emph{p-valores}
obtenidos de alguna forma: \uncover<+->{bootstrap (muy lenta),} \uncover<+->{Bonferroni (muy conservadora) }\uncover<+->{o \emph{false discovery rate} (FDR).}

\pause{}

Si se testean $k$ diferentes hipótesis, La \emph{“tasa de falso
descubrimiento”} es la proporción esperada de hipótesis incorrectamente
rechazadas. Si se testea $k$ veces la misma hipótesis, la FDR coincide
con el nivel de significación del test. 

\pause{}

En particular, si $p_{\left(1\right)}\leq\dots\leq p_{\left(k\right)}$son
los $k$p-valores (ordenados), el nivel de un test que rechace $H_{0}$cuando
$\alpha\geq\inf\left\{ \frac{k}{i}p_{\left(i\right)},i=1,\dots,k\right\} $
es, a lo sumo, de $\alpha$.

\pause{}
\begin{alertblock}{FDR vs. Bonferroni}

Bonferroni rechaza con nivel $\alpha$ cuando $p_{\left(1\right)}\leq$$\alpha/k$,
en cuyo caso FDR también rechaza. En la práctica, FDR suele ser mucho
menos conservador que Bonferroni.
\end{alertblock}
\end{frame}
%
\begin{frame}{Consideraciones prácticas: ¿cuántas direcciones tomar?}

Resta definir el número de proyecciones. $k=30$ es más que suficientemente
conservador, y en algunos casos bastará con $k\in\left\{ 1,\dots,5\right\} $.
Es importante considerar que a mayor $k$,

\pause{}
\begin{itemize}
\item Bajo $H_{0},$ más conservador resulta el test y menor será la probabilidad
de rechazo
\end{itemize}

\pause{}
\begin{itemize}
\item bajo una alternativa fija $H_{1}$, mayor será la probabilidad de
rechazo
\end{itemize}

\pause{}

Este último efecto puede deberse a que mientras más direcciones aleatorias
se tomen, más altas son las chances de que la alternativa correlacione
fuertemente con alguna de ellas y este efecto contrarreste lo conservador
del procedimiento. Por ello, en \cite{Cuesta 2010} recomiendan tomar
$k=\min\left\{ 30,n\right\} .$
\end{frame}

\subsection{En R: usando \texttt{fda.usc::fanova.RPm}}
\begin{frame}[<+->][fragile]{En R: usando \texttt{fda.usc::fanova.RPm}}



\begin{knitrout}\footnotesize
\definecolor{shadecolor}{rgb}{0.969, 0.969, 0.969}\color{fgcolor}\begin{kframe}
\begin{alltt}
\hlkwd{library}\hlstd{(fda.usc)}
\end{alltt}
\end{kframe}
\end{knitrout}

\pause{}

\begin{knitrout}\footnotesize
\definecolor{shadecolor}{rgb}{0.969, 0.969, 0.969}\color{fgcolor}\begin{kframe}
\begin{alltt}
\hlstd{X} \hlkwb{<-} \hlkwd{cbind}\hlstd{(fda}\hlopt{::}\hlstd{growth}\hlopt{$}\hlstd{hgtm, fda}\hlopt{::}\hlstd{growth}\hlopt{$}\hlstd{hgtf)}
\hlstd{grilla} \hlkwb{<-} \hlstd{fda}\hlopt{::}\hlstd{growth}\hlopt{$}\hlstd{age}
\hlstd{factores} \hlkwb{<-} \hlkwd{data.frame}\hlstd{(}
    \hlkwc{varon}\hlstd{=}\hlkwd{as.factor}\hlstd{(}\hlkwd{startsWith}\hlstd{(}\hlkwd{colnames}\hlstd{(X) ,} \hlstr{"boy"}\hlstd{))}
\hlstd{)}
\end{alltt}
\end{kframe}
\end{knitrout}

\pause{}

\begin{knitrout}\footnotesize
\definecolor{shadecolor}{rgb}{0.969, 0.969, 0.969}\color{fgcolor}\begin{kframe}
\begin{alltt}
\hlstd{X.fd} \hlkwb{<-} \hlkwd{fdata}\hlstd{(}\hlkwc{mdata}\hlstd{=}\hlkwd{t}\hlstd{(X),} \hlkwc{argvals}\hlstd{=grilla)}
\hlstd{test} \hlkwb{<-} \hlkwd{fanova.RPm}\hlstd{(X.fd,} \hlopt{~} \hlstd{varon, factores,} \hlkwc{RP}\hlstd{=}\hlnum{1}\hlstd{)}
\hlstd{test}\hlopt{$}\hlstd{p.FDR}
\end{alltt}
\begin{verbatim}
##            varon
## RP1 2.038658e-11
\end{verbatim}
\end{kframe}
\end{knitrout}

\end{frame}
\begin{frame}[shrink=30]{Referencias}

\appendix
\begin{thebibliography}{References}
\bibitem[Cuesta 2007]{Cuesta 2007}Cuesta-Albertos, J. A., del Barrio,
E., Fraiman, R., \& Matrán, C. (2007). The random projection method
in goodness of fit for functional data. Computational Statistics \&
Data Analysis, 51(10), 4814-4831.

\bibitem[Cuevas 2007]{Cuevas 2007}Cuevas, A., Febrero, M., \& Fraiman,
R. (2007). Robust estimation and classification for functional data
via projection-based depth notions. Computational Statistics, 22(3),
481-496.

\bibitem[Cuesta 2010]{Cuesta 2010}Cuesta-Albertos, J. A., \& Febrero-Bande,
M. (2010). A simple multiway ANOVA for functional data. Test, 19(3),
537-557.

\bibitem[Patilea 2012]{Patilea 2012}Patilea, V., Sanchez-Sellero,
C., \& Saumard, M. (2012). Projection-based nonparametric goodness-of-fit
testing with functional covariates. arXiv preprint arXiv:1205.5578.

\bibitem[Garcia 2014]{Garcia 2014}García-Portugués, E., González-Manteiga,
W., \& Febrero-Bande, M. (2014). A goodness-of-fit test for the functional
linear model with scalar response. Journal of Computational and Graphical
Statistics, 23(3), 761-778. 

\bibitem[Cuesta 2017]{Cuesta 2017}Cuesta-Albertos, J. A., Febrero-Bande,
M., \& de la Fuente, M. O. (2017). The $\hbox{DD}^{G}$-classifier
in the functional setting. Test, 26(1), 119-142.

\bibitem[Cuesta 2019]{Cuesta 2019}Cuesta-Albertos, J. A., García-Portugués,
E., Febrero-Bande, M., \& González-Manteiga, W. (2019). Goodness-of-fit
tests for the functional linear model based on randomly projected
empirical processes. The Annals of Statistics, 47(1), 439-467.

\bibitem[fda.usc]{fda.usc} Librería fda.usc
\end{thebibliography}
\end{frame}
%
\begin{frame}[standout]

{\huge{}¡Gracias!}{\huge\par}

\pause{}

{\scriptsize{}\alert{{\scriptsize{}\href{https://github.com/memfcen-amateur/proyecciones-aleatorias/tree/master}{github.com/memfcen-amateur/proyecciones-aleatorias}}}}{\scriptsize\par}

\end{frame}

\end{document}
